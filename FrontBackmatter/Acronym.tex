\begin{acronym}[SPARQL]
% A
% B
% C
% D
\acro{DFG}{Deutsche Forschungsgemeinschaft}
% E
% F
% G
% H
\acro{HCI}{Human-Computer Interaction}
\acro{HITO}{Health IT-Ontology}
\acro{HTML}{HyperText Markup Language}
% I
% J
% K
\acro{KIS}{Krankenhausinformationssystem}
\acrodefplural{KIS}[KIS]{Krankenhausinformationssysteme}
% L
% M
\acro{MCI}{Mensch-Computer Interaktion}
\acro{MMI}{Mensch-Maschine Interaktion}
% N
% O
\acro{OWL}{Web Ontology Language}
% P
\acro{PDF}{Portable Document Format}
% Q
% R
\acro{RDF}{Resource Description Framework}
% S
\acro{SPARQL}{SPARQL Protocol And RDF Query Language}
% T
% U
\acro{UI}{User Interface}
\acro{UX}{User Experience}
% V
% W
\acro{WWW}{World Wide Web}
% X
\acro{XHTML}{Extensible HyperText Markup Language}
% Y
% Z
\end{acronym}
\acused{URL}% Has its own paragraph in the preliminaries.
