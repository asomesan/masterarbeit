%*******************************************************
% Summary
%*******************************************************
%\pdfbookmark[1]{Zusammenfassung}{summary}
\chapter{Zusammenfassung}
%\addcontentsline{toc}{chapter}{Zusammenfassung}

Im Rahmen des HITO-Projektes wurde eine Ontologie und Wissensbasis zur systematischen Beschreibung von Anwendungssystemen und Softwareprodukten der medizinischen Informatik entwickelt.
Ziel dieser Arbeit ist, ein neues Visualisierungskonzept für die Webanwendung zu entwickeln, um die Ontologie und Wissensbasis erkunden zu können.
Für die Entwicklung des neues Visualisierungskonzeptes wurde die User-Centered Design Methode, auch nutzerzentrierter Entwicklungsprozess genannt, angewendet.
Der Entwicklungsprozess besteht aus vier Schritten: Analyse, Konzeption, Design und Evaluation.
In jedem Schritt des nutzerzentrierten Entwicklungsprozesses wurden unterschiedliche Techniken angewendet.
Die vier Schritte wurden in der festgelegten Reihenfolge abgearbeitet.
Im Zuge der Analyse wurde eine Befragung durchgeführt und Personas definiert.
Somit wurden die Anforderungen an die neue Webanwendung definiert und die Zielgruppen identifiziert.
Im zweiten Schritt, sprich der Konzeption, wurden im Laufe von zwei Iterationen die Wireframes der Webanwendung entwickelt.
Danach wurden im dritten Schritt des nutzerzentrierten Entwicklungsprozesses die finalen Screens der Webanwendung gestaltet und zu einem klickbaren Prototypen umgewandelt.
Der gesamte Prozess endete mit einem Usability Review.
Hierfür wurden die zehn Heuristiken nach Nielsen verwendet.
Im Rahmen der Evaluation wurde auch ein Vergleich zwischen der aktuellen Lösung und dem neuen Visualisierungskonzept der Webanwendung vollzogen.
Aus dem Vergleich hat sich herausgestellt, dass das neue Visualisierungskonzept besser an die Zielgruppen angepasst ist und die Bedingungen der Heuristiken besser erfüllt.

\vfill
