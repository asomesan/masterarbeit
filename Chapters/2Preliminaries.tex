%*****************************************
\chapter{Grundlagen}\label{ch:preliminaries}
%*****************************************

Das Grundlagenkapitel soll den Stand der Forschung erläutern und mit Literatur belegen.
Auf diesem Kapitel bauen die Erkenntnisse der Arbeit auf.
Gerade in den Grundlagen wird man häufig Quellen benennen, aus denen die Aussagen letztlich stammen.
Im Kapitel \enquote{Literaturverzeichnis} dieser Vorlage wird beschrieben, wie eine Quellenangabe zu erfolgen hat.

Da sich die Medizinische Informatik mit der Lösung medizinischer Probleme befasst, sollen hier auch die Hintergründe des medizinischen Problems so dargestellt und erläutert werden, dass sie auch für Leser der Arbeit, die nicht Mediziner sind, verständlich sind.

In diesem Kapitel werden auch die Methoden erläutert, die zur Lösung der Probleme eingesetzt wurden.
Stellen Sie sicher, dass hier alle, aber auch nur die Grundlagen und Methoden erläutert werden, die in der Arbeit verwendet wurden.
Stellen Sie im weiteren Text der Arbeit auch sicher, dass der Leser erkennen kann, wie Sie unter Verwendung der Methoden zu Ihren Ergebnissen gekommen sind.
So sollten z.B. Modellierungsmethoden nur verwendet werden, wenn die Modelle nachvollziehbar dazu genutzt werden, die Ergebnisse zu erzielen.

Bedenken Sie, dass Sie diese Arbeit zum Abschluss eines umfangreichen Studiums schreiben, das vor allem dazu diente, sie mit einem reichen Methodenrepertoire auszustatten.
Wählen Sie aus den Methoden, die Sie gelernt haben, aus, benennen Sie die Methoden korrekt und wenden Sie sie an! Aber gehen sie auch kritisch mit dem um, was Ihnen gelehrt wurde.
Wenn Sie feststellen, dass gelehrte Methoden ungeeignet sind, diskutieren Sie dies und suchen passendere Methoden! Wenn Sie Methoden benötigen, die nicht gelehrt wurden, suchen Sie nach passenden Methoden oder -- wenn Sie nicht fündig werden -- entwickeln Sie die für Ihr Problem passende Methode 

\section{Medizinische Informatik}\label{sec:mi}
Definitionen: KIS, IS, System
\begin{definition}[KIS]
  This is a definition.
\end{definition}

\subsection{Daten, Information, Wissen}
\subsection{Krankenhausinformationssysteme}
\subsection{Management von Krankenhausinformationssysteme}
Definitionen: CIO, Würfel, 
\subsection{Health IT Ontology} 

\section{Semantic Web}\label{sec:sw}
\subsection{World Wide Web}
\subsection{Linked Open Data}
\subsection{Linked Data Visualisierung}  
\subsection{Ontologien}
\subsection{Knowledge Base}
\subsection{OWL - Web Ontology Language}
\subsection{RDF} 
\subsection{SPARQL} 

\section{User Experience und User Interface Design}\label{sec:ux}
