%************************************************
\chapter{Einleitung}\label{ch:introduction}

Die vorliegende Arbeit beschäftigt sich mit der Suche in einer Ontologie und Wissensbasis von Beschreibungen von Anwendungssystemen und Softwareprodukten der Gesundheitsversorgung.

Dieses einleitende Kapitel schafft einen Einblick in der Thematik der Abschlussarbeit. Die bevorstehende Probleme werden genannt und das daraus resultierende Nutzen wird erläutert. Weiterhin werden die Ziele und die dazugehörenden Aufgaben vorgestellt. Abschließend wird einen Überblick über den Aufbau der vorliegenden Arbeit geschafft.

\section{Gegenstand}\label{sec:gegenstand}
Es wird geschätzt, das Google 63.000 Suchanfragen pro Sekunde verarbeitet. Das Internet ermöglicht den Menschen einen Zugang zu einer unvorstellbaren Menge an Wissen. Es gibt unzählige Webseiten wo man sich sowohl über verschiedenste Themen informieren kann als auch Webseiten wo man sich zu bestimmte Themen austauschen kann. Jeder Nutzer kann unterwegs mit seinem Smartphone in ein paar Sekunden ein bestimmtes Thema nachschlagen. Ausschlaggebend bei der Suche ist aber, wie die Daten im Hintergrund strukturiert sind und wie der Nutzer gezielt danach suchen kann.

Auch in der medizinischen Informatik steht eine große Menge an Wissen zur Verfügung in Form von Studien, wissenschaftliche Arbeiten, Bücher, Berichte u.v.m. In diesem Fall ist es von Vorteil, wenn das Wissen aus unterschiedlichen Quellen strukturiert aufbereitet wird, um es dann dem Leser für die Analyse zur Verfügung stellen zu können. Linked Data ist in diesem Fall der Standard für die Aufbereitung und Veröffentlichung von strukturierte Daten im World Wide Web. 

Die Suche auf einer Webseite ist ein mächtiges Werkzeug, das sicherstellen soll, dass Nutzer in der Lage sind, effizient Produkte oder andere Inhalte zu finden. Bei der Gestaltung der Suche sollen verschiedene Aspekte betrachtet werden, wie zum Beispiel Verhaltensmuster, Usability und sogar die Ergebnisseite, um die Suche so robust und nutzerfreundlich zu gestalten.

Health IT Ontology (HITO) ist ein Projekt, das der Beschreibung von Applikationssystemen und Softwareprodukten im Bereich der Gesundheitsversorgung dient. Hier haben CIOs die Möglichkeit, nach Open-Source-Softwareprodukten für ihre Krankenhäuser zu suchen. Die Plattform soll den Austausch von Wissen und Erfahrungen erleichtern und bei der Entscheidungsfindung in der Planung und Steuerung von Krankenhausinformationssystemen unterstützen.

%Die vorliegende Arbeit beschäftigt sich mit der Suche in einer Ontologie und Wissensbasis . Diese basiert aktuell auf das Multi-Facetted-Search Modell und unterstützt nicht die Suche nach Zusammenhänge. Es soll untersucht werden, ob die aktuelle Lösung oder eine andere Suchmethode am besten geeignet sind. 

\section{Problemstellung}\label{sec:problemstellung}

Der Suchprozess in einer Wissensbasis besteht aus mehreren Elemente wie zum Beispiel die Suchquerie, die Datenbasis, die Art und Weise in der die Suchergebnisse dargestellt werden und wie die Nutzer mit der Suchmaske interagieren.  Aktuell besteht im HITO die Möglichkeit über zwei separate Suchmasken entweder nach Studien oder nach Softwareprodukte zu suchen. Beide Suchmasken basieren auf dem Multi-Facetted-Suchmodell und die Ergbenisseite besteht aus einer Liste der zutreffenden Studien oder Softwareprodukten.

Auf dem Markt existieren bereits unterschiedliche Lösungen, um bestimmte Use Cases im Bereich der Visualisierung von Linked Data abzudecken. Es stellt sich jetzt die Frage, welche der existierenden Lösungen im HITO eingesetzt werden können und ob es die Möglichkeit gibt unterschiedliche Lösungen zu einer Masterlösung zusammenzuführen.

Für die vorliegende Arbeit lassen sich die folgende Probleme ableiten:

\begin{itemize}
\item Problem P1: Aktuell ist es unklar, was der genaue Informationsbedarf der Nutzer ist und welche Abfragen auf das aktuelle Datenmodell möglich sind.
\item Problem P2: Es fehlt eine Übersicht der aktuellen Lösungen, die passend für die im Projekt beschriebenen Use-Cases eingesetzt werden können.
\item Problem P3: Die bestehende Lösung wurde nicht ausführlich untersucht hinsichtlich der Benutzererfahrung.
\end{itemize}

\section{Motivation}\label{sec:motivation}

Das HITO-Projekt kann zu unterschiedliche Zwecke eingesetzt werden. Aus diesem Grund soll die Benutzererfahrung im Prozess der Suche verbessert werden. Somit kann das Projekt nicht nur von Spezialisten im Bereich der medizinischen Informatik benutzt werden, sondern zum Beispiel auch von Studenten zur Begriffserklärung im Rahmen von Vorlesungen oder von Krankenhauspersonal im Rahmen einer Weiterbildung. Des Weiteren profitieren auch die Vorgesetzten der Abteilung von Informationsmanagement in einem Krankenhaus davon, da diese basierend auf die Studien und Berichte informierte Entscheidungen bezüglich der eingesetzten Software treffen kann. Somit lässt sich die Suche nach neuer Software einfacher und nicht so zeitintensiv gestalten.

Um all diese Erwartungen erfüllen zu können, soll der Suchprozess so konzipiert werden, dass jeder Nutzer sich durch die Daten der Ontologie unkompliziert und zielgerichtet navigieren kann.

\section{Zielsetzung}\label{sec:zielsetzung}

Basierend auf die im Abschnitt \nameref{sec:problemstellung} genannten Probleme, lassen sich die folgenden Ziele ableiten:

\begin{itemize}
\item Ziel Z1: Das erste Ziel der Arbeit ist die Ermittlung des Informationsbedarfs der Nutzer basierend auf den Beispielfragen und des Projektantrags. 
\item Ziel Z2: Das zweite Ziel der Arbeit ist die systematische Aufteilung des Informationsbedarfs in Kategorien im Hinblick auf existierende Techniken.
\item Ziel Z3: Das dritte Ziel der Arbeit ist das Konzept und Design des Suchprozesses zu erstellen. Dieser wird in Form eines Prototyps für Testzwecke zur Verfügung gestellt.
\end{itemize}

\section{Aufgabenstellung}\label{sec:aufgabenstellung}

Um die im Abschnitt \nameref{sec:zielsetzung} gesetzten Ziele zu erreichen müssen folgende Aufgaben erfüllt werden:

\begin{itemize}
\item Aufgaben zu Ziel Z1:
	\begin{itemize}
	\item Aufgabe A1.1: Beschreibung der Use Cases im HITO-Projekt.
	\item Aufgabe A1.2: Filterung der Fragen anhand der beschriebenen Use Cases.
	\end{itemize}
\item Aufgaben zu Ziel Z2:
	\begin{itemize}
	\item Aufgabe A2.1: Beschreibung der Use Cases im Rahmen der Linked Data Visualisierung.
	\item Aufgabe A2.2: Vergleich der Fragen mit den Use Cases.
	\item Aufgabe A2.3: Vorstellung der aus dem Vergleich resultierenden Lösungen.
	\end{itemize}
\item Aufgaben zu Ziel Z3:
	\begin{itemize}
	\item Aufgabe A3.1: Konzept- und Interface-Design anhand der im Ziel Z2 erreichten Ergebnissen.
	\item Aufgabe A3.2: Suchmaske in Form eines Prototyps für Testzwecke zur Verfügung stellen.
	\end{itemize}
\end{itemize}

\section{Aufbau der Arbeit}\label{sec:aufbau}

Die vorliegende Masterarbeit besteht aus acht Kapiteln. Die Kapiteln vier bis sechs stellen den Hauptbeitrag der Arbeit dar. \newline

\textbf{Kapitel 1} stellt die Probleme, Motivation und Ziele der Arbeit dar und schafft einen Überblick über die Struktur der Arbeit. \newline

\textbf{Kapitel 2} schafft einen Überblick über die theoretischen Grundlagen der medizinischen Informatik, des Managements von Krankenhausinformationssystemen (KIS), des Semantic Webs und des User-Experience und User Interface Designs. Dabei werden die relevanten Definitionen im Hinblick auf die vorliegende Arbeit erläutert.  \newline

\textbf{Kapitel 3} untersucht und beschreibt bereits existierende Lösungen für die Visualisierung von Linked Data. Dabei wird genauer die aktuell eingesetzte Lösung im HITO-Projekt betrachtet und die Lösungen aus dem Buch "Linked Data Visualization". \newline

\textbf{Kapitel 4} beinhaltet eine kurze Beschreibung der Vorhaben, die zur Lösung der im Abschnitt \nameref{sec:problemstellung} genannten Probleme dient. \newline

\textbf{Kapitel 5} schafft einen ausführlichen Einblick in die Ausführung der Lösung.  \newline

\textbf{Kapitel 6} stellt die Ergebnisse der Arbeit dar. Dabei wird die resultierende Suchmaske anhand von Bildern erläutert. \newline

\textbf{Kapitel 7} bewertet die Ergebnisse der Arbeit in Form einer kritischen Diskussion. \newline

\textbf{Kapitel 8} beendet die Arbeit mit einer Zusammenfassung und schafft einen Einblick in zukünftige Entwicklungen und Herausforderungen. Es werden noch mal die im Abschnitt \nameref{sec:zielsetzung} genannten Ziele aufgegriffen und dann mit den Ergebnissen der Arbeit verglichen.



