%************************************************
\chapter{Einleitung}\label{ch:introduction}

Die vorliegende Arbeit beschäftigt sich mit der Optimierung der Benutzererfahrung im Bereich der Suche in einer Ontologie und Wissensbasis von Anwendungssystemen und Softwareprodukten der Gesundheitsversorgung.

Dieses einleitende Kapitel schafft einen Einblick in der Thematik der Abschlussarbeit. 
Die bevorstehenden Probleme werden genannt und das daraus resultierende Nutzen wird erläutert. 
Weiterhin werden die Ziele und die dazugehörenden Aufgaben vorgestellt. 
Abschließend wird ein Überblick über den Aufbau der Arbeit geschafft.

\section{Gegenstand}\label{sec:gegenstand}

%\begin{center}
%\textit{"... the Semantic Web is a jungle - a rich mass of interconnected information, without any roadmap, index, or guidance."}
%\end{center}

%\begin{flushright}
%\textit{ \textendash{} \citep{allemang}} \newline
%\end{flushright}

%Web für Mensch als Endnutzer ausgerichtet aber für Maschinen nicht. 
%Ziel einer Suche sind mehr inhaltsbasierte Ergebnisse anstatt von stichwortbasierte Ergebnisse.
%Das \ac{WWW} ermöglicht den Menschen einen Zugang zu einer unvorstellbaren Menge an Wissen und Informationen. 
%Es gibt unzählige Webseiten, wo man sich sowohl über verschiedenste Themen informieren kann als auch Webseiten, wo man sich zu bestimmte Themen austauschen kann. 
%Jeder Nutzer kann unterwegs mit seinem Smartphone in ein paar Sekunden ein bestimmtes Thema nachschlagen. 

%Die Art und Weise, in der Inhalte im \ac{WWW} aufbereitet sind, ist für Menschen als Endnutzer geeignet. 
%Dabei reicht es die Informationen mittels \ac{HTML} aufzubereiten und dem Menschen zur Verfügung zu stellen. 
%Metadaten sind hierfür nicht nötig \citep[vgl.]{pellegrinix}. 
%Maschinen hingegen können daraus kein Wissen generieren oder die vorliegenden Informationen in Zusammenhang mit anderen Informationen setzen \citep[vgl.]{hitzler}. 
%Diese brauchen Metadaten. 
%Im Bereich des Semantic Webs werden Metadaten mittels \ac{RDF} modelliert. Im \cref{sec:sw} wird das Semantic Web ausführlich präsentiert und die zugrundeliegenden Technologien werden vorgestellt.

%Wie das Zitat am Anfang des Abschnitts schon verdeutlicht, ist das Semantic Web eine reiche Menge an Informationen die miteinander vernetzt sind. 
%Um das ganze Netz an Informationen navigieren und Zusammenhänge selber erfassen zu können, muss ein Nutzer über bestimmte Kenntnisse verfügen.
%Ein Neueinsteiger muss mit erheblichen Einarbeitungszeiten rechnen.

Der Beitrag der medizinischen Informatik im Gesundheitswesen soll eine bestmögliche Gesundheitsversorgung unterstützen.
Aus diesem Grund ist es wichtig, dass Anwendungssysteme einwandfrei laufen und immer verfügbar sind und Softwareprodukte die Erwartungen der Stakeholder erfüllen.

\ac{HITO} ist ein Forschungsprojekt unterstützt von der \ac{DFG}.
Ziel des Projektes ist die Entwicklung einer Ontologie zur systematischen Beschreibung von Anwendungssystemen und Softwareprodukten in der medizinischen Informatik.
Dabei kann der Austausch von Wissen und Erfahrungen erleichtert werden, die Kooperation bei der Planung und Steuerung von \acp{KIS} kann unterstützt werden und evidenzbasierte Entscheidungen im Rahmen des Informationsmanagements werden unterstützt.

Zielgruppe des HITO-Projekts sind unter anderem Mitarbeiter eines Krankenhauses im Bereich des Informationsmanagements. 
HITO soll ihnen ein einfacheres Suchen nach geeigneten Softwareprodukten und eine informierte Entscheidungsfindung ermöglichen. 
Aus diesem Grund ist eine gute User Experience von Vorteil. 

"Properly designed technology that is centered on the user experience (UX) can make a positive difference in many domains."
\newline  User-Centered Design Ansatz nutzen bei der Konzepterstellung, führt zu nützliche Tools, die einfach zu bedienen sind und ihren Potenzial durch eine gute Benutzererfahrung erreichen können.


User Experience Bedeutung 

\section{Problemstellung}\label{sec:problemstellung}

Im aktuellen Zustand kann man die Wissensbasis über drei unterschiedliche Optionen erkunden.
Die drei Optionen sind die folgenden:
\begin{enumerate}
\item \textbf{Facettierte Suche}
\newline Die facettierte Suchmaske ist aktuell die Hauptlösung zur Erkundung der Wissensbasis. 
Diese filterbasierte Suchmethode ermöglicht eine gezielte Suche nach med.floss-Softwareprodukten indem man sich die gewünschten Merkmale aus einer Liste von Features, Enterprise Functions, Anwendungssysteme, Sprachen, (Clients) und Datenbanktypen auswählt. 
\item \textbf {RDF Browser}
\newline Der RDF-Browser ist eine zusätzliche Möglichkeit, um die Wissensbasis zu erkunden und setzt technisches Wissen voraus.
Hierbei kann man keine genaue Suche durchführen.
Die Wissensbasis lässt sich mittels dem RDF-Browser nur durch Durchklicken erkunden. 
\item \textbf {SPARQL Query Editor }
\newline Der SPRAQL Query Editor ist eine weitere zusätzliche Möglichkeit, um die Wissensbasis zu erkunden, die auf der Projektseite angeboten wird. 
Diese setzt Wissen der SPARQL-Abfragesprache und des Vokabulars der Ontologie vor.
Hierbei werden Abfragen auf die Datenbasis der Ontologie gestartet, die dann die zutreffenden Ergebnisse liefert.
\end{enumerate} 

Die facettierte Suche liegt im Fokus der vorliegenden Arbeit. 
Aus diesem Grund werden die anderen zwei Optionen im weiteren Verlauf der Arbeit nicht betrachtet.

Für die vorliegende Arbeit lässt sich das folgende Problem formulieren:

\begin{itemize}
\item Problem P1: Die aktuelle Hauptlösung zur Erkundung der Wissensbasis wird von den Stakeholdern als nicht ausreichend bewertet.
Bestimmte nützliche Besonderheiten können nicht erfasst werden, wie zum Beispiel die Darstellung der Zusammenhänge aus dem Metamodell der Ontologie. 
Andere Lösungen zur Erkundung einer Wissensbasis wurden nicht untersucht.
\end{itemize}

\section{Motivation}\label{sec:motivation}

Das HITO-Projekt kann zu unterschiedliche Zwecke eingesetzt werden. Aus diesem Grund soll die Benutzererfahrung im Navigationsprozess der Wissensbasis verbessert werden. Somit kann das Projekt von allen Stakeholdern zielgerichtet benutzt werden. Zum Beispiel können Studenten HITO zur Begriffserklärung im Rahmen von Vorlesungen benutzen oder HITO kann als Unterstützung vom Krankenhauspersonal im Rahmen einer Weiterbildung eingesetzt werden. Des Weiteren profitieren auch die Vorgesetzten der Abteilung von Informationsmanagement in einem Krankenhaus davon, da diese basierend auf die Studien und Berichte informierte Entscheidungen bezüglich der eingesetzten Software treffen kann. Somit lässt sich die Suche nach neuer Software einfacher und zeiteffizient gestalten.

Um all diese Erwartungen erfüllen zu können, soll die Navigation so gestaltet sein, dass jeder Nutzer sich durch die Daten der Wissensbasis unkompliziert und zielgerichtet navigieren kann.

\section{Zielsetzung}\label{sec:zielsetzung}

Basierend auf das im Abschnitt \nameref{sec:problemstellung} genanntes Problem, lässt sich der folgende Ziel ableiten:

\begin{itemize}
\item Ziel Z1: Ziel der Arbeit ist die Erstellung eines Konzept- und Interface-Designs des Suchprozesses. 
Dabei soll beachtet werden, dass beide Nutzergruppen, Experten und Nichtexperten, ohne einen großen Zeitaufwand die Wissensbasis navigieren können.
\end{itemize}

\section{Aufgabenstellung}\label{sec:aufgabenstellung}

Um das im Abschnitt \nameref{sec:zielsetzung} gesetzte Ziel der Arbeit zu erreichen, müssen folgende Aufgaben erfüllt werden:

\begin{itemize}
	\item Aufgabe A1: Ermittlung des Informationsbedarfs der Nutzer.
		\begin{itemize}
		\item Aufgabe A1.1: Sammlung möglicher Nutzersuchanfragen.
		\item Aufgabe A1.2: Die Nutzersuchanfragen werden den Use Cases aus dem HITO-Projektantrag zugeordnet.
		\end{itemize}
	\item Aufgabe A2: Die systematische Aufteilung des Informationsbedarfs im Hinblick auf existierende Techniken.
		\begin{itemize}
		\item Aufgabe A2.1: Zuordnung der Nutzersuchanfragen zu den Use Cases aus \citet{linkeddatavisualization}.
		\item Aufgabe A2.2: Auswahl einer geeigneten Menge an Werkzeugen, die die in A2.1 identifizierten Use Cases abdecken.
		\end{itemize}
	\item Aufgabe A3: Erstellung eines Konzept- und Interface-Design.
		\begin{itemize}
		\item Aufgabe A3.1: Konzept- und Interface-Design basierend auf den Ergebnissen der Aufgabe 2.
		\item Aufgabe A3.2: Bereitstellung eines Mockups für Testzwecke.
		\end{itemize}
\end{itemize}

\section{Aufbau der Arbeit}\label{sec:aufbau}

Die vorliegende Masterarbeit besteht aus acht Kapiteln. Die Kapiteln vier bis sechs stellen den Hauptbeitrag der Arbeit dar. \newline

\cref{ch:introduction} stellt die Probleme, Motivation und Ziele der Arbeit dar und schafft einen Überblick über die Struktur der Arbeit. \newline

\textbf{Kapitel 2} schafft einen Überblick über die theoretischen Grundlagen der medizinischen Informatik, des Managements von Krankenhausinformationssystemen (KIS), des Semantic Webs und des User-Experience und User Interface Designs. Dabei werden die relevanten Definitionen im Hinblick auf die vorliegende Arbeit erläutert.  \newline

\textbf{Kapitel 3} untersucht und stellt bereits existierende Lösungen für die Visualisierung von Linked Data vor. Dabei wird genauer die aktuell eingesetzte Lösung im HITO-Projekt betrachtet und die Lösungen aus dem Buch "Linked Data Visualization". \newline

\textbf{Kapitel 4} beinhaltet eine kurze Beschreibung der Vorhaben, die zur Lösung der im Abschnitt \nameref{sec:problemstellung} genannten Probleme dient. \newline

\textbf{Kapitel 5} schafft einen ausführlichen Einblick in die Ausführung der Lösung.  \newline

\textbf{Kapitel 6} stellt die Ergebnisse der Arbeit dar. Dabei wird die resultierende Suchmaske anhand von Bildern erläutert. \newline

\textbf{Kapitel 7} bewertet die Ergebnisse der Arbeit in Form einer kritischen Diskussion. \newline

\textbf{Kapitel 8} beendet die Arbeit mit einer Zusammenfassung und schafft einen Einblick in zukünftige Entwicklungen und Herausforderungen. Es werden noch mal die im Abschnitt \nameref{sec:zielsetzung} genannten Ziele aufgegriffen und dann mit den Ergebnissen der Arbeit verglichen.



