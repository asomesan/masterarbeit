%************************************************
\chapter{Einleitung}\label{ch:introduction}

Die vorliegende Arbeit beschäftigt sich mit der Suche in einer Ontologie und Wissensbasis von Anwendungssystemen und Softwareprodukten der Gesundheitsversorgung.

Dieses einleitende Kapitel schafft einen Einblick in der Thematik der Abschlussarbeit. 
Die bevorstehenden Probleme werden genannt und das daraus resultierende Nutzen wird erläutert. 
Weiterhin werden die Ziele und die dazugehörenden Aufgaben vorgestellt. 
Abschließend wird einen Überblick über den Aufbau der vorliegenden Arbeit geschafft.

\section{Gegenstand}\label{sec:gegenstand}

\begin{center}
\textit{"... the Semantic Web is a jungle - a rich mass of interconnected information, without any roadmap, index, or guidance."}
\end{center}

\begin{flushright}
\textit{ \textendash{} \citep{allemang}} \newline
\end{flushright}

%Web für Mensch als Endnutzer ausgerichtet aber für Maschinen nicht. 
%Ziel einer Suche sind mehr inhaltsbasierte Ergebnisse anstatt von stichwortbasierte Ergebnisse.
Das \ac{WWW} ermöglicht den Menschen einen Zugang zu einer unvorstellbaren Menge an Wissen und Informationen. 
Es gibt unzählige Webseiten, wo man sich sowohl über verschiedenste Themen informieren kann als auch Webseiten, wo man sich zu bestimmte Themen austauschen kann. 
Jeder Nutzer kann unterwegs mit seinem Smartphone in ein paar Sekunden ein bestimmtes Thema nachschlagen. 

Die Art und Weise, in der Inhalte im \ac{WWW} aufbereitet sind, ist für Menschen als Endnutzer geeignet. 
Dabei reicht es die Informationen mittels \ac{HTML} aufzubereiten und dem Menschen zur Verfügung zu stellen. 
Metadaten sind hierfür nicht nötig \citep[vgl.]{pellegrinix}. 
Maschinen hingegen können daraus kein Wissen generieren oder die vorliegenden Informationen in Zusammenhang mit anderen Informationen setzen \citep[vgl.]{hitzler}. 
Diese brauchen Metadaten. 
Im Bereich des Semantic Webs werden Metadaten mittels \ac{RDF} modelliert. Im \cref{sec:sw} wird das Semantic Web ausführlich präsentiert und die zugrundeliegenden Technologien werden vorgestellt.

Wie das Zitat am Anfang des Abschnitts schon verdeutlicht, ist das Semantic Web eine reiche Menge an Informationen die miteinander vernetzt sind. 
Um das ganze Netz an Informationen navigieren zu können, muss ein Nutzer über bestimmte Kenntnisse verfügen.
Ein Neueinsteiger muss mit erheblichen Einarbeitungszeiten rechnen.

\section{Problemstellung}\label{sec:problemstellung}

Zwei Anforderungen die Nutzer erfüllen müssen, um das Semantic Web benutzen zu können, sind die Beherrschung einer formalen Abfragesprache wie zum Beispiel \ac{SPARQL} und das Besitzen von Wissen über den Wortschatz der Wissensbasis, die sie abfragen wollen \citep[vgl.]{hoffner2017survey}. 
Das sind Eigenschaften, die in der Regel Experten im Bereich von Semantic Web besitzen. 
Zielgruppe des HITO-Projekts sind aber unter anderem Mitarbeiter eines Krankenhauses im Bereich des Informationsmanagements.
Diese besitzen keine Erfahrung im Bereich des Semantic Webs und der dazugehörenden Technologien.

Im aktuellen Zustand ist die Navigation in der Wissensbasis des HITO-Projekts fragmentiert. 
Man hat den RDF Browser und den SPARQL Query Editor. 
Diese können nur von Nutzer, die sich mit der Terminologie der Wissensbasis und SPARQL auskennen, genutzt werden. 
Zusätzlich hat man auch die Facettierte Navigation. 
Diese ist im Grunde genommen, von allen Nutzer anwendbar. 
Bei der Implementierung wurde die Benutzererfahrung nicht beachtet und es liegen nicht genug Informationen vor, dass die Facettierte Navigation die geeignete Methode ist.
Somit lässt sich das Projekt von Nichtexperten nur durch einer sehr zeitintensiven Einarbeitung nutzen. 

Für die vorliegende Arbeit lässt sich das folgende Problem formulieren:

\begin{itemize}
\item Problem P1: Der IST-Zustand des Suchprozesses besteht aus drei Fragmenten. 
Zwei benötigen Wissen und Technologien, das nur Experten im Bereich von Semantic Web besitzen und das dritte Fragment wurde nicht ausreichend hinsichtlich Benutzererfahrung untersucht.
Zusammenhänge lassen sich durch die drei Fragmente auch nicht erfassen.
\end{itemize}

\section{Motivation}\label{sec:motivation}

Das HITO-Projekt kann zu unterschiedliche Zwecke eingesetzt werden. Aus diesem Grund soll die Benutzererfahrung im Prozess der Suche verbessert werden. Somit kann das Projekt nicht nur von Spezialisten im Bereich des Semantic Webs benutzt werden, sondern zum Beispiel auch von Studenten zur Begriffserklärung im Rahmen von Vorlesungen oder von Krankenhauspersonal im Rahmen einer Weiterbildung. Des Weiteren profitieren auch die Vorgesetzten der Abteilung von Informationsmanagement in einem Krankenhaus davon, da diese basierend auf die Studien und Berichte informierte Entscheidungen bezüglich der eingesetzten Software treffen kann. Somit lässt sich die Suche nach neuer Software einfacher und zeiteffizient gestalten.

Um all diese Erwartungen erfüllen zu können, soll der Suchprozess so konzipiert werden, dass jeder Nutzer sich durch die Daten der Ontologie unkompliziert und zielgerichtet navigieren kann.

\section{Zielsetzung}\label{sec:zielsetzung}

Basierend auf das im Abschnitt \nameref{sec:problemstellung} genanntes Problem, lässt sich der folgende Ziel ableiten:

\begin{itemize}
\item Ziel Z1: Ziel der Arbeit ist die Erstellung eines Konzept- und Interface-Designs des Suchprozesses. 
Dabei soll beachtet werden, dass beide Nutzergruppen, Experten und Nichtexperten, ohne einen großen Zeitaufwand die Wissensbasis navigieren können.
\end{itemize}

\section{Aufgabenstellung}\label{sec:aufgabenstellung}

Um das im Abschnitt \nameref{sec:zielsetzung} gesetzte Ziel der Arbeit zu erreichen, müssen folgende Aufgaben erfüllt werden:

\begin{itemize}
	\item Aufgabe A1: Ermittlung des Informationsbedarfs der Nutzer.
		\begin{itemize}
		\item Aufgabe A1.1: Sammlung möglicher Nutzersuchanfragen.
		\item Aufgabe A1.2: Die Nutzersuchanfragen werden den Use Cases aus dem HITO-Projektantrag zugeordnet.
		\end{itemize}
	\item Aufgabe A2: Die systematische Aufteilung des Informationsbedarfs im Hinblick auf existierende Techniken.
		\begin{itemize}
		\item Aufgabe A2.1: Zuordnung der Nutzersuchanfragen zu den Use Cases aus \citet{linkeddatavisualization}.
		\item Aufgabe A2.2: Auswahl einer geeigneten Menge an Werkzeugen, die die in A2.1 identifizierten Use Cases abdecken.
		\end{itemize}
	\item Aufgabe A3: Erstellung eines Konzept- und Interface-Design.
		\begin{itemize}
		\item Aufgabe A3.1: Konzept- und Interface-Design basierend auf den Ergebnissen der Aufgabe 2.
		\item Aufgabe A3.2: Bereitstellung eines Mockups für Testzwecke.
		\end{itemize}
\end{itemize}

\section{Aufbau der Arbeit}\label{sec:aufbau}

Die vorliegende Masterarbeit besteht aus acht Kapiteln. Die Kapiteln vier bis sechs stellen den Hauptbeitrag der Arbeit dar. \newline

\cref{ch:introduction} stellt die Probleme, Motivation und Ziele der Arbeit dar und schafft einen Überblick über die Struktur der Arbeit. \newline

\textbf{Kapitel 2} schafft einen Überblick über die theoretischen Grundlagen der medizinischen Informatik, des Managements von Krankenhausinformationssystemen (KIS), des Semantic Webs und des User-Experience und User Interface Designs. Dabei werden die relevanten Definitionen im Hinblick auf die vorliegende Arbeit erläutert.  \newline

\textbf{Kapitel 3} untersucht und stellt bereits existierende Lösungen für die Visualisierung von Linked Data vor. Dabei wird genauer die aktuell eingesetzte Lösung im HITO-Projekt betrachtet und die Lösungen aus dem Buch "Linked Data Visualization". \newline

\textbf{Kapitel 4} beinhaltet eine kurze Beschreibung der Vorhaben, die zur Lösung der im Abschnitt \nameref{sec:problemstellung} genannten Probleme dient. \newline

\textbf{Kapitel 5} schafft einen ausführlichen Einblick in die Ausführung der Lösung.  \newline

\textbf{Kapitel 6} stellt die Ergebnisse der Arbeit dar. Dabei wird die resultierende Suchmaske anhand von Bildern erläutert. \newline

\textbf{Kapitel 7} bewertet die Ergebnisse der Arbeit in Form einer kritischen Diskussion. \newline

\textbf{Kapitel 8} beendet die Arbeit mit einer Zusammenfassung und schafft einen Einblick in zukünftige Entwicklungen und Herausforderungen. Es werden noch mal die im Abschnitt \nameref{sec:zielsetzung} genannten Ziele aufgegriffen und dann mit den Ergebnissen der Arbeit verglichen.



