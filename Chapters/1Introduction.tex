%************************************************
\chapter{Einleitung}\label{ch:introduction}

Es wird geschätzt, das Google 63.000 Suchanfragen pro Sekunde verarbeitet. Das Internet ermöglicht den Menschen einen Zugang zu einer unvorstellbaren Menge an Wissen. Es gibt unzählige Webseiten wo man sich sowohl über verschiedenste Themen informieren kann als auch Webseiten wo man sich zu bestimmte Themen austauschen kann. Jeder Nutzer kann unterwegs mit seinem Smartphone in ein paar Sekunden ein bestimmtes Thema nachschlagen. Ausschlaggebend bei der Suche ist aber, wie die Daten im Hintergrund strukturiert sind und wie diese dem Nutzer präsentiert werden.

Auch in der medizinischen Informatik steht eine große Menge an Wissen zur Verfügung in Form von Studien, wissenschaftliche Arbeiten, Bücher, Berichte u.v.m. In diesem Fall ist es von Vorteil, wenn das Wissen aus unterschiedlichen Quellen strukturiert aufbereitet wird, um es dann dem Leser für die Analyse zur Verfügung stellen zu können. Linked Data ist in diesem Fall der Standard für die Aufbereitung und Veröffentlichung von strukturierte Daten im World Wide Web. 

Die Suche auf einer Webseite ist ein mächtiges Werkzeug, das sicherstellen soll, dass Nutzer in der Lage sind, effizient Produkte oder andere Inhalte zu finden. Bei der Gestaltung der Suche sollen verschiedene Aspekte betrachtet werden, wie zum Beispiel Verhaltensmuster, Usability und sogar die Ergebnisseite, um die Suche so robust und nutzerfreundlich zu gestalten.

In diesem Kapitel wird beschrieben, um was es in der vorliegenden Masterarbeit handelt, welche Probleme vorliegen und welche Aufgaben erfüllt werden müssen, um die gesetzten Ziele zu erreichen. Des Weiteren werden auch die Gründe genannt, warum die Lösung der Probleme von Vorteil ist. Abschließend wird einen Überblick über den Aufbau der vorliegenden Arbeit geschafft.


\section{Gegenstand}\label{sec:gegenstand}

Health IT Ontology (HITO) ist ein Projekt, das der Beschreibung von Applikationssystemen und Softwareprodukten im Bereich der Gesundheitsversorgung dient. Hier haben CIOs die Möglichkeit, nach Open-Source-Softwareprodukten für ihre Krankenhäuser zu suchen. Die Plattform soll den Austausch von Wissen und Erfahrungen erleichtern und bei der Entscheidungsfindung in der Planung und Steuerung von Krankenhausinformationssystemen unterstützen.

In der vorliegenden Arbeit wird die Suche in der Ontologie genauer betrachtet. Diese basiert aktuell auf das Multi-Facetted-Search Modell und unterstützt nicht die Suche nach Zusammenhänge. Es soll untersucht werden, ob die aktuelle Lösung oder eine andere Suchmethode am besten geeignet sind. 

\section{Problemstellung}\label{sec:problemstellung}

Es existieren bereits unterschiedliche Lösungen für bestimmte Use Cases im Bereich der Visualisierung von Linked Data. Was aktuell fehlt ist eine Lösung, die die unterschiedlichen Use Cases in einer einzigen Anwendung abdeckt, eine Lösung die es ermöglicht nach Wissen, Zusammenhänge zwischen unterschiedliche Paper,  Daher lassen sich für die vorliegende Arbeit folgende Probleme ableiten:

\begin{itemize}
\item Problem P1: Die aktuelle Suche beruht auf einer Multi-Facetted-Suche und es gibt nicht genug Informationen, die bestätigen, dass die aktuelle Lösung am besten geeignet ist.
\item Problem P2: Die Suchmaske besteht aus mehreren Auswahlmöglichkeiten und verknüpft mit der darin enthaltenden Ergebnisliste kann es unübersichtlich werden und zu einer Fehlinterpretation führen.
\end{itemize}

\section{Motivation}\label{sec:motivation}



\section{Zielsetzung}\label{sec:zielsetzung}

Basierend auf die im Abschnitt \nameref{sec:problemstellung} genannten Probleme, lassen sich die folgenden Ziele ableiten:

\begin{itemize}
\item Ziel Z1: Das erste Ziel der Arbeit ist die Ermittlung des Informationsbedarfs der Nutzer basierend auf den Beispielfragen und des Projektantrags. 
\item Ziel Z2: Das zweite Ziel der Arbeit ist die systematische Aufteilung des Informationsbedarfs in Kategorien im Hinblick auf existierende Techniken.
\item Ziel Z3: Das dritte Ziel der Arbeit ist die Erstellung der Suchmaske, die in Form eines Clickdummys zur Verfügung gestellt werden soll.
\end{itemize}

%\begin{itemize}
%\item Ziel Z1: Das erste Ziel der Arbeit ist die Analyse von unterschiedlichen Suchmethoden, um durch das Ergebnis der Analyse die geeignete Suchmethode zu finden.
%\item Ziel Z2: Das zweite Ziel der Arbeit ist die Gestaltung der Suchmaske, basierend auf die Ergebnisse der im Ziel Z1 durchgeführten Analyse.
%\end{itemize}

%\begin{itemize}
%\item Welche Ergebnisse werden mit dieser Abschlussarbeit angestrebt und welche der o.\,g. Probleme sollen damit jeweils gelöst werden?
%\end{itemize}
%Bitte jedes Ziel kurz oder ggf. mit Stichworten beschreiben:
%\begin{itemize}
%\item Ziel(e)/angestrebte(s) Ergebnis(se) zur Lösung von Problem P1:
%	\begin{itemize}
%	\item Ziel Z1.1: ....
%	\item Ziel Z1.2: ....
%	\end{itemize}
%\end{itemize}

\section{Aufgabenstellung}\label{sec:aufgabenstellung}

Um die im Abschnitt \nameref{sec:zielsetzung} gesetzten Ziele zu erreichen müssen folgende Aufgaben erfüllt werden:

\begin{itemize}
\item Aufgaben zu Ziel Z1:
	\begin{itemize}
	\item Aufgabe A1.1: Beschreibung der Use Cases im HITO-Projekt.
	\item Aufgabe A1.2: Filterung der Fragen anhand der beschriebenen Use Cases.
	\end{itemize}
\item Aufgaben zu Ziel Z2:
	\begin{itemize}
	\item Aufgabe A2.1: Beschreibung der Use Cases im Rahmen der Linked Data Visualisierung.
	\item Aufgabe A2.2: Vergleich der Fragen mit den Use Cases.
	\item Aufgabe A2.3: Vorstellung der aus dem Vergleich resultierenden Lösungen.
	\end{itemize}
\item Aufgaben zu Ziel Z3:
	\begin{itemize}
	\item Aufgabe A3.1: Konzept- und Interface-Design anhand der im Ziel Z2 erreichten Ergebnissen.
	\item Aufgabe A3.2: Suchmaske in Form eines Prototyps für Testzwecke zur Verfügung stellen.
	\end{itemize}
\end{itemize}

\section{Aufbau der Arbeit}\label{sec:aufbau}

Die vorliegende Masterarbeit besteht aus sieben Kapiteln.Die Kapiteln vier bis sechs stellen den Hauptbeitrag der Arbeit dar. \newline

\textbf{Kapitel 1} stellt die Probleme, Motivation und Ziele der Arbeit dar und schafft einen Überblick über die Struktur der Arbeit. \newline

\textbf{Kapitel 2} stellt die Probleme, Motivation und Ziele der Arbeit dar und schafft einen Überblick über die Struktur der Arbeit. \newline

\textbf{Kapitel 2} stellt die Probleme, Motivation und Ziele der Arbeit dar und schafft einen Überblick über die Struktur der Arbeit. \newline

\textbf{Kapitel 4} stellt die Probleme, Motivation und Ziele der Arbeit dar und schafft einen Überblick über die Struktur der Arbeit. \newline

\textbf{Kapitel 5} stellt die Probleme, Motivation und Ziele der Arbeit dar und schafft einen Überblick über die Struktur der Arbeit. \newline

\textbf{Kapitel 6} stellt die Probleme, Motivation und Ziele der Arbeit dar und schafft einen Überblick über die Struktur der Arbeit. \newline

\textbf{Kapitel 7} stellt die Probleme, Motivation und Ziele der Arbeit dar und schafft einen Überblick über die Struktur der Arbeit. \newline

Durch die vorliegende Arbeit soll herausgefunden werden, welche Suchmethode sich für die Softwaresuche im Rahmen des HITO-Projekts am besten eignet. Nach dem einleitenden Kapitel, in dem Gegenstand, Problem, Motivation, Zielsetzung und Aufgabenstellung erläutert werden, folgt die Erläuterung der theoretischen Grundlagen, worauf das weitere Vorhaben basiert. Dabei werden Begriffe der medizinischen Informatik erklärt und Bereiche und deren Aufgaben, wie zum Beispiel das Management von Informationssystemen präsentiert. Des Weiteren wird in dem Grundlagenkapitel die User Experience betrachtet. Abschließend werden dann unterschiedliche Suchmethoden präsentiert, die im Analyseprozess betrachtet werden.

Nach dem Grundlagenkapitel folgt die Präsentation des aktuellen Forschungsstands. Dabei werden ähnliche Projekte kurz vorgestellt und analysiert. Im Kapitel vier werden die Lösungsansätze zu den in Kapitel \nameref{sec:problemstellung} genannten Probleme vorgestellt. Anschließend dazu wird eine ausführliche Beschreibung der Lösungsfindung stattfinden. Dabei werden die Lösungen der Probleme einzeln präsentiert, um dann im Kapitel sechs die daraus resultierenden Ergebnisse vorzustellen.

Die Arbeit schließt mit einer kritischen Bewertung der Ergebnisse in Form einer Diskussion und einer Zusammenfassung der Ergebnisse. In der Zusammenfassung werden noch mal die im Kapitel \nameref{sec:zielsetzung} genannten Ziele aufgegriffen und dann mit den Ergebnissen der Arbeit verglichen.


