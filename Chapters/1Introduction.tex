%************************************************
\chapter{Einleitung}\label{ch:introduction}

Die vorliegende Arbeit beschäftigt sich mit der Optimierung der Benutzererfahrung und der Benutzeroberfläche im Bereich der Datenerkundung in einer Ontologie und Wissensbasis von Anwendungssystemen und Softwareprodukten der Gesundheitsversorgung.

Dieses einleitende Kapitel schafft einen Einblick in der Thematik der Abschlussarbeit. 
Das bevorstehende Problem wird genannt und das daraus resultierende Nutzen wird erläutert. 
Weiterhin wird das Ziel der Arbeit vorgestellt und im Anschluss dazu die daraus resultierenden Aufgaben. 
Abschließend wird ein Überblick über den Aufbau der Arbeit geschafft.

\section{Gegenstand}\label{sec:gegenstand}

\begin{center}
\textit{"... ontologies are used on the information source and the business logic layer, and thus hidden “under the hood”."}
\end{center}

\begin{flushright}
\textit{ \textendash{} \citep{paulheim2010ontology}} \newline
\end{flushright}

Ontologien stellen eine Methode für das Wissensmanagement und die Erfassung von Beziehungen dar.
Sie werden für unterschiedliche Zwecke eingesetzt, beispielsweise zur Integration von Daten aus verschiedenen Quellen und Systemen.
Die Ontologie und dessen Modell werden aber in den meisten Fällen nicht in der Benutzeroberfläche integriert, sondern stehen nur beispielsweise als Informationsquelle im Hintergrund. \citep[vgl.]{paulheim2010ontology}

Doch die Integration der Ontologie in der Benutzeroberfläche kann von Vorteil sein. 
Nutzer können sich dann schnell ein Gesamtbild schaffen und mit einem kurzen Einblick verstehen, welche Klassen das Datenset beinhaltet und welche Beziehungen diese verbinden. \citep[vgl.]{linkeddatavisualization}

\ac{HITO} ist eine Ontologie zur systematischen Beschreibung von Anwendungssystemen und Softwareprodukten in der medizinischen Informatik.
Diese beinhaltet Begriffe, die zur Beschreibung von Softwareprodukten benutzt werden können, aber auch Studien, die diese Softwareprodukte bewerten.
Die Ontologie entstand im Rahmen eines Forschungsprojektes, unterstützt von der Deutschen Forschungsgemeinschaft (\acs{DFG}).
Die Intention des HITO-Projek-tes ist ein einfacher Austausch von Wissen und Erfahrungen zu ermöglichen, eine bessere Kooperation im Rahmen der Planung und Steuerung von \acp{KIS} zu unterstützen und eine evidenzbasierte Entscheidungsfindung im Rahmen des Gesundheitsinformationsmanagements zu erlauben.

\section{Problemstellung}\label{sec:problemstellung}

%Im aktuellen Zustand kann man die Wissensbasis über drei unterschiedliche Optionen erkunden.
%Die drei Optionen sind die folgenden:
%\begin{enumerate}
%\item \textbf{Facettierte Suche}
%\newline Die facettierte Suchmaske ist aktuell die Hauptlösung zur Erkundung der Wissensbasis. 
%Diese filterbasierte Suchmethode ermöglicht eine gezielte Suche nach med.floss-Softwareprodukten indem man sich die gewünschten Merkmale aus einer Liste von Features, Enterprise Functions, Anwendungssysteme, Sprachen, (Clients) und Datenbanktypen auswählt. 
%\item \textbf {RDF Browser}
%\newline Der RDF-Browser ist eine zusätzliche Möglichkeit, um die Wissensbasis zu erkunden und setzt technisches Wissen voraus.
%Hierbei kann man keine genaue Suche durchführen.
%Die Wissensbasis lässt sich mittels dem RDF-Browser nur durch Durchklicken erkunden. 
%\item \textbf {SPARQL Query Editor }
%\newline Der SPRAQL Query Editor ist eine weitere zusätzliche Möglichkeit, um die Wissensbasis zu erkunden, die auf der Projektseite angeboten wird. 
%Diese setzt Wissen der SPARQL-Abfragesprache und des Vokabulars der Ontologie vor.
%Hierbei werden Abfragen auf die Datenbasis der Ontologie gestartet, die dann die zutreffenden Ergebnisse liefert.
%\end{enumerate} 
%Die facettierte Suche liegt im Fokus der vorliegenden Arbeit. 
%Aus diesem Grund werden die anderen zwei Optionen im weiteren Verlauf der Arbeit nicht betrachtet.

Im aktuellen Zustand lässt sich die Ontologie über eine facettierte Suchmaske erkunden. 
Dabei werden gewünschte Optionen ausgewählt und man erhält als Ergebnis eine Liste der zutreffenden Objekte.
Diese Methode ermöglicht zwar die Suche in der Ontologie, aber schafft keinen Einblick in der Art und Weise in der die Objekte miteinander in Relation stehen.
Die Struktur der Ontologie lässt sich somit nicht erkunden.
Eine visuelle Darstellung der Daten schafft Raum für Exploration und zufällige Entdeckungen. 
Nutzer die keine technische Kompetenzen besitzen können durch eine Visualisierung der Ontologie das Konzept und dessen Inhalte besser begreifen. \citep[vgl.]{linkeddatavisualization} \newline
\newline
Das Problem, was im Rahmen der vorliegenden Abschlussarbeit gelöst werden soll, ist das folgende:

\begin{itemize}
\item Problem P1: Die aktuelle Lösung zur Erkundung der Ontologie wird von den Stakeholdern als nicht ausreichend bewertet.
Bestimmte nützliche Besonderheiten können nicht erfasst werden, wie zum Beispiel die Darstellung der in der Ontologie enthaltenen Relationen oder eine explorative Erkundung der Daten. 
Andere Lösungen zur Visualisierung einer Ontologie wurden nicht untersucht.
\end{itemize}

\section{Motivation}\label{sec:motivation}

Das HITO-Projekt kann zu unterschiedliche Zwecke eingesetzt werden. Aus diesem Grund soll für die Benutzeroberfläche ein neues Visualisierungskonzept erstellt werden. Somit kann das Projekt von allen Stakeholdern zielgerichtet benutzt werden. Zum Beispiel können Studenten HITO zur Begriffserklärung im Rahmen von Vorlesungen benutzen oder HITO kann als Unterstützung vom Krankenhauspersonal im Rahmen einer Weiterbildung eingesetzt werden. Des Weiteren profitieren auch die Vorgesetzten der Abteilung von Informationsmanagement in einem Krankenhaus davon, da diese basierend auf die Studien und Berichte informierte Entscheidungen bezüglich der eingesetzten Software treffen kann. Somit lässt sich die Suche nach neuer Software einfacher und zeiteffizient gestalten.

\section{Zielsetzung}\label{sec:zielsetzung}

Basierend auf das im Abschnitt \nameref{sec:problemstellung} genanntes Problem, lässt sich der folgende Ziel ableiten:

\begin{itemize}
\item Ziel Z1: Ziel der Arbeit ist die konzeptuelle Erstellung eines Visualisierungskonzeptes der Health IT Ontology und dessen Instanzen.
Die Klassen und Relationen der Ontologie sollen einfach und verständlich dargestellt werden und die Nutzer sollen sich ohne großen Zeitaufwand sich durch das Wissen navigieren können.
\end{itemize}

\section{Aufgabenstellung}\label{sec:aufgabenstellung}

Um das im Abschnitt \nameref{sec:zielsetzung} gesetzte Ziel der Arbeit zu erreichen, müssen folgende Aufgaben erfüllt werden:

\begin{itemize}
	\item Aufgabe A1: Ermittlung des Informationsbedarfs der Nutzer.
		\begin{itemize}
		\item Aufgabe A1.1: Sammlung möglicher Nutzersuchanfragen.
		\item Aufgabe A1.2: Die Nutzersuchanfragen werden den Use Cases aus dem HITO-Projektantrag zugeordnet.
		\end{itemize}
	\item Aufgabe A2: Die systematische Aufteilung des Informationsbedarfs im Hinblick auf existierende Techniken.
		\begin{itemize}
		\item Aufgabe A2.1: Zuordnung der Nutzersuchanfragen zu den Use Cases aus \citet{linkeddatavisualization}.
		\item Aufgabe A2.2: Auswahl einer geeigneten Menge an Visualisierungsmethoden und Werkezugen, die die in A2.1 identifizierten Use Cases abdecken.
		\end{itemize}
	\item Aufgabe A3: Erstellung eines Konzeptes für die Benutzeroberfläche.
		\begin{itemize}
		\item Aufgabe A3.1: Konzept der Visualisierung und Design der Benutzeroberfläche basierend auf den Ergebnissen der Aufgabe 2.
		\item Aufgabe A3.2: Bereitstellung eines Mockups für Testzwecke.
		\end{itemize}
\end{itemize}

\section{Aufbau der Arbeit}\label{sec:aufbau}

Die vorliegende Masterarbeit besteht aus acht Kapiteln. Die Kapiteln vier bis sechs stellen den Hauptbeitrag der Arbeit dar. \newline

\cref{ch:introduction} stellt die Probleme, Motivation und Ziele der Arbeit dar und schafft einen Überblick über die Struktur der Arbeit. \newline

\textbf{Kapitel 2} schafft einen Überblick über die theoretischen Grundlagen der medizinischen Informatik, des Managements von Krankenhausinformationssystemen (KIS), des Semantic Webs und des User-Experience und User Interface Designs. Dabei werden die relevanten Definitionen im Hinblick auf die vorliegende Arbeit erläutert.  \newline

\textbf{Kapitel 3} untersucht und stellt bereits existierende Lösungen für die Visualisierung von Linked Data vor. Dabei wird genauer die aktuell eingesetzte Lösung im HITO-Projekt betrachtet und die Lösungen aus dem Buch "Linked Data Visualization". \newline

\textbf{Kapitel 4} beinhaltet eine kurze Beschreibung der Vorhaben, die zur Lösung der im Abschnitt \nameref{sec:problemstellung} genannten Probleme dient. \newline

\textbf{Kapitel 5} schafft einen ausführlichen Einblick in die Ausführung der Lösung.  \newline

\textbf{Kapitel 6} stellt die Ergebnisse der Arbeit dar. Dabei wird die resultierende Suchmaske anhand von Bildern erläutert. \newline

\textbf{Kapitel 7} bewertet die Ergebnisse der Arbeit in Form einer kritischen Diskussion. \newline

\textbf{Kapitel 8} beendet die Arbeit mit einer Zusammenfassung und schafft einen Einblick in zukünftige Entwicklungen und Herausforderungen. Es werden noch mal die im Abschnitt \nameref{sec:zielsetzung} genannten Ziele aufgegriffen und dann mit den Ergebnissen der Arbeit verglichen.



