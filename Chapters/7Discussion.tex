%*****************************************
\chapter{Diskussion und Ausblick}\label{ch:discussion}
%*****************************************
Durch Anwendung der User-Centered Design Methodik wurde ein neues Visualisierungskonzept für HITO entwickelt.
Es wurden \newline Mockups für die wichtigsten Bereiche der Webanwendung erstellt.
Im Folgenden werden nun zunächst die Ergebnisse der Arbeit kritisch bewertet.

Die User-Centered Design Methodik ist ein agiler Entwicklungsprozess, wodurch eine Anwendung im Laufe mehrerer Iterationen optimiert wird.
Die Anzahl der Iterationen ist im Rahmen der vorliegenden Arbeit durch den Umfang der Bearbeitungsdauer begrenzt.
Folglich können weitere Iterationen dazu beitragen, die Webanwendung an die Ansprüche weiterer Zielgruppen anzupassen.
So könnte das User Centered Design dafür verwendet werden, damit HITO auch den Anforderungen und Bedürfnissen von Expert*innen im Bereich des Semantic Webs gerecht wird.

Die angewendeten Techniken des nutzerzentrierten Entwicklungsprozesses sind, bis auf das Usability Review zur Durchführung der Evaluation, die meist erwähntesten Techniken in der Literatur. \citep[vgl.]{salinas_2020}
Hier ist kritisch anzumerken, dass die ausgewählte Evaluationstechnik aus zeitlichen Gründen ausgewählt wurde.
Die Evaluation wurde anhand der zehn Heuristiken nach Nielsen durchgeführt.
Um weitere Schwächen und Verbesserungsmöglichkeiten des Visualisierungskonzeptes zu identifzieren empfiehlt sich (der weitaus umfangreichere) User Acceptance Test. Ein User Acceptance Test würde eine praxisorientierte Evaluation auf Basis der Erfahrung von realen Nutzer*innen erlauben.
Somit können Schwächen und Verbesserungsmöglichkeiten im Rahmen einer Evaluation mit echten Nutzer*innen identifiziert werden.

Die Arbeitsergebnisse der vorliegenden Arbeit stellen Ausgangspunkt für die Entwicklung der neuen Webanwendung von HITO dar.
Die erstellten Mockups sind in der GitHub Repository \footnote{https://github.com/asomesan/masterarbeit} zu finden und können durch die CSS-Code Export Funktion die Entwicklung erleichtern.
Aus der bisher existierenden Lösung, die für die Erkundung der Ontologie und der Wissensbasis entwickelt wurden, können der RDF-Browser RickView und der SPARQL-Endpunkt weiterverwendet werden, um Experten*innen im Bereich von Semantic Web einen Einblick in die Tiefen der Ontologie und der Wissensbasis weiterhin zu ermöglichen.
Des Weiteren soll für die zukünftige Anwendung überlegt werden, welche Daten angezeigt werden.
Betrachtet man ein Softwareprodukt, werden dort beispielsweise die Unternehmensaufgaben und die Features direkt von der Herstellerbeschreibung übernommen und angezeigt.
Es soll überlegt werden, ob man an dieser Stelle beispielsweise die klassifizierten Einträge aus der Wissensbasis zum Vergleich anzeigt.

Der neue Stand der Visualisierung stellt in Form der Mockups und des klickbaren Prototypen ein solides Fundament für die Weiterentwicklung von HITO dar. 
Kurzfristig bedarf es einer tiefergehenden Evaluation der Mockups und einer Anpassung entsprechend der im Zuge dessen erhobenen Verbesserungsvorschläge.

