%*****************************************
\chapter{Diskussion und Ausblick}\label{ch:discussion}
%*****************************************
Durch Anwendung der User-Centered Design Methodik wurde ein neues Visualisierungskonzept für HITO entwickelt.
Es wurden Mockups für die wichtigsten Bereiche der Website erstellt.
Im Folgenden werden nun zunächst die Ergebnisse der Arbeit kritisch bewertet.

Die User-Centered Design Methodik ist ein agiles Entwicklungsprozess.
Demnach kann eine Anwendung im Laufe mehrerer Iterationen Zustande kommen.
Die Anzahl der Iterationen ist im Rahmen der vorliegenden Arbeit durch den Umfang der Bearbeitungsdauer begrenzt.
Somit ist schlusszufolgern, dass in der Zukunft die Erweiterung der Anwendung von weiteren Iterationen profitieren kann.
Es soll auch erwähnt werden, dass die User-Centered Design Methodik angewendet werden kann, um die Website für weitere Zielgruppen zu erweitern.
Die Anwendung kann zum Beispiel in der Zukunft auch Anforderung und Bedürfnisse von Experten im Bereich des Semantic Webs erfüllen.

Die Arbeitsergebnisse sollen im nächsten Schritt als Ausgangspunkt für die Entwicklung der neuen Website von HITO dienen.
Somit können ausführliche Usability-Acceptance Tests mit Nutzer*innen durchgeführt werden, um Schwächen und Verbesserungspotenziale zu identifizieren.

- Evaluation wurde nur anhand eines Usability Experten Reviews gemacht, daher sollen noch weitere ausführlichere Tests mit Anwender der drei Zielgruppen durchgeführt werden \\
- Herkunft der Daten aktuell nur auf Citation Ebene, evtl. in der Zukunft auch die Classified Begriffe anzeigen

