%*****************************************
\chapter{Lösungsansatz}\label{ch:approach}
%*****************************************
Dieses Kapitel konzentriert sich auf die Anwendung der User-Centered Design Methodik zur Lösung der Problemstellung der Arbeit und präsentiert die ausgewählten Techniken in jedem Schritt des nutzerzentrierten Entwicklungsprozesses.
Die Abbildung \ref{fig:approach} stellt die einzelnen Schritte zur Lösung des Problems dar.
Es sollte dabei erwähnt werden, dass sich die \ac{UCD}-Methodik in den einzelnen Schritten an unterschiedliche Techniken bedient.
Die angewendeten Techniken wurden mit Hilfe des Papers \enquote{A Systematic Review of User-Centered Design Techniques} von \citet{salinas_2020} ausgewählt.

\begin{figure}[h]
	\centering
    	\includegraphics[width=\textwidth]{Images/Ansatz}
   	\caption{Prozessschritte}
   	\label{fig:approach}
\end{figure}

\section{Analyse}

Der erste Schritt im Prozess der nutzerzentrierten Entwicklung ist die Durchführung einer Analyse.
Das Ziel der Analyse ist es den Nutzungskontext zu verstehen und beschreiben.
Auf Basis des Nutzungskontextes werden genaue Anforderungen an das Design definiert.
Die nachfolgenden Unterkapitel setzen sich mit den zwei ausgewählten Methoden zur Analyse im \ac{UCD}-Prozess auseinander.

\subsection{Befragung}\label{ch:interview}

Befragungen werden durchgeführt, um die Nutzeranforderungen zu erheben.
Ziel der Befragung ist sowohl die Datensammlung über mögliche Nutzer*innen als auch Informationen über ihre Anforderungen an die neue Darstellung der Wissensbasis und Einschätzungen zu der aktuellen Anwendung zu sammeln. 
Im Rahmen der vorliegenden Arbeit wird eine teilweise nicht standardisierte Befragung durchgeführt.
Es werden sowohl offene Fragen, als auch geschlossene Fragen gestellt.
Die offenen Fragen können in diesem Fall zu neuen Erkenntnissen und Zusammenhängen führen.
Durch die geschlossenen Fragen, wird die aktuelle Lösung bewertet.
Die Ergebnisse der Befragung sollen als Basis der Definition von Personas im nächsten Schritt dienen.

\subsection{Personas}

Im Zuge des Entwicklungsprozesses sollte man die Benutzenden und ihre Bedürfnisse stets im Blick behalten.
Aus diesem Grund werden mögliche Nutzer*innen und deren Nutzverhalten in Form von Personas definiert.
Die gesammelten Daten aus der im Unterkapitel \ref{ch:interview} durchführten Befragung dienen zur Definition der Personas.
Personas sind fiktive Nutzer der Anwendung, die repräsentativ für die Zielgruppe stehen.
Die ausgearbeiteten Personas und deren Bedürfnisse werden im Laufe des nutzerzentrierten Entwicklungsprozesses im Mittelpunkt stehen.

\section{Konzeption}\label{sec:concept}

Im zweiten Schritt des \ac{UCD}-Prozesses werden die Nutzungsanforderungen spezifiziert.
Hierbei wird die zugrundeliegende Informationsarchitektur und der User-Flow entwickelt.
Für diesen Schritt der nutzerzentrierten Entwicklung wurden Wireframes als Methode ausgewählt.
Wireframes konzentrieren sich ausschließlich auf den Aufbau der Seiten, die dargestellten Inhalte und die Interaktionsmöglichkeiten der Nutzer.

\section{Design}

Der dritte Schritt im \ac{UCD}-Prozess ist die Entwicklung der Gestaltungslösung.
Anhand der durchgeführten Analyse und des erstellten Konzeptes soll die Visualisierung der Anwendung verfeinert werden.
In diesem Schritt werden Methoden wie Mockups und Prototypen eingesetzt.
Im Folgenden werden die zwei Methoden präsentiert und deren Auswahl wird begründet.

\subsection{Mockups}\label{sec:mockup}

Bei dieser Methode werden die im Unterkapitel \ref{sec:concept} erstellten Wireframes zu detaillierten Screens ausgearbeitet.
Im Vergleich zu den Wireframes werden bei den Mockups auch Designelemente verwendet wie zum Beispiel Farben, Typografie oder Bilder.

\subsection{Prototypen}

Mit Hilfe von einem Protoytp werden einzelne Teile der Anwendung simuliert.
Als Grundlage werden in diesem Schritt die im Unterkapitel \ref{sec:mockup} resultierende Mockups verwendet.

\section{Evaluation}

Der vierte und letzte Schritt der nutzerzentrierten Entwicklung sieht die Evaluation der entwickelten Lösung hervor.
Dabei wird die erarbeitete Gestaltungslösung im Rahmen eines Usability Reviews getestet.
Das Usability Review kann durch zwei Ansätze durchgeführt werden und zwar der aufgabenbasierte Ansatz und der richtlinienbasierte Ansatz.
Im vorliegenden Fall wird der richtlinienbasierte Ansatz durchgeführt.
Dabei wird das neue User Interface Design anhand der zehn Heuristiken nach Nielsen begutachtet.


